\documentclass[demo]{beamer}
\usepackage{movie15}

\usetheme{Rochester}
\usecolortheme{dolphin}
\usenavigationsymbolstemplate{}
\setbeamertemplate{footline}[frame number]

%%TiKz
\usepackage{tikz}
\usetikzlibrary{positioning}
\newenvironment{psmallmatrix}
  {\left(\begin{smallmatrix}}
  {\end{smallmatrix}\right)}


\AtBeginSection[]
{
\begin{frame}<beamer>{Table of Contents}
\tableofcontents[currentsection,currentsubsection, 
    hideothersubsections, 
    sectionstyle=show/shaded,
]
\end{frame}
}

\title{Bayesian Non-Parametric Methods for Epidemic Models}
\author{Rowland Seymour \\ Supervisors: Philip O'Neill and Theodore Kypraios}
\institute{University of Nottingham}
\date{Nottingham, July 2017}

\begin{document}

\frame{\titlepage}

\begin{frame}
\frametitle{Table of Contents}
\tableofcontents
\end{frame}


\section{Epidemic Models}
\begin{frame}
	\frametitle{SIR Models}
	\begin{figure}[h]
\begin{center}
\begin{tikzpicture}[squarenode/.style={rectangle, draw=black, very thick, minimum size=1cm},]
\node[squarenode] (susceptible) 								{$S$};
\node[squarenode] (infected) 	[right =of susceptible] 		{$I$}; 
\node[squarenode] (recovered) 	[right =of infected] 		{$R$}; 
\draw[thick, ->] (susceptible.east) -- node[above] {} (infected.west);
\draw[thick, ->] (infected.east) 	-- node[above] {} (recovered.west);
\end{tikzpicture}	
\end{center}
\caption{A typical SIR diagram.}
\label{fig: Basic SIR Diagram}
\end{figure}

Given infection and removal times, can we work out the rate that individuals move between classes? \\
\vspace{1em}
In the standard epidemic model, we assume infections occur according to a Poisson process with rate $\beta_0 S_tI_t$ and the removals occur according to a Poisson process with rate $\gamma I_t$. \\
\vspace{1em}
In epidemic inference, we try to estimate $\beta_0$ and $\gamma$.



\end{frame}

\begin{frame}
%% Motivation - Why Epidemic Modelling
	\frametitle{The Infection Rate}
	In the standard epidemic model, the infection rate is $\beta_0 S_t I_t$. \\
	\vspace{1em}
	\textbf{Problems:}\\
	\begin{itemize}
	\setlength\itemsep{1em}
		\item Is this the functional form of the infection rate?
		\item Do other covariates, such as time or distance, affect the infection rate?
	\end{itemize}
	\textbf{Solutions:}\\
		\begin{itemize}
	\setlength\itemsep{1em}
		\item Use non-parametric models. 
		\item Inference for a heterogeneously mixing model.
	\end{itemize}
\end{frame}

\section{Non-Parametric Models for Epidemics}
\begin{frame}
%% Motivation - Non-Parametric Models
	\frametitle{Non-Parametric Models}
	Using a non-parametric inference has the following advantages:
	\vspace{1em}
	\begin{itemize}
	\setlength\itemsep{1em}
		\item Let the data speak for itself
		\item Remove bias from model choice
		\item Flexible choice of model
	\end{itemize}
\end{frame}


\begin{frame}
	\frametitle{Assumptions and Ideas}
%% Non-Parametric Models - Assumptions and Ideas

\textbf{Infections:}\\
Instead of assuming the infection rate is $\beta_0 S_t I_t$, we assume the infection rate can be modelled by an inhomogeneous Poisson process with rate $\beta$, where 

$$
\beta = f(t) \quad \hbox{or} \quad \beta = f(x, y) \quad \hbox{or} \quad \beta = f(S_t, I_t).
$$



\textbf{Removals:}\\
The infected individuals remain so for a period, which has some given distribution, for example:

$$
\hbox{Exp}(\gamma) \quad \hbox{or} \quad \hbox{Gamma}(\lambda, \gamma).
$$

\end{frame}


\begin{frame}
%% Non-Parametric Models - Bayesian Inference
	\frametitle{Bayesian Inference for Non-Parametric Models}
\textbf{Problems:} \\
We have an infinite set of functions for $\beta$. How do we place a prior distribution on $\beta$ and infer the function in a reasonable amount of time?
\vspace{1em}

\textbf{Solutions:}\\
We use Gaussian processes to estimate the infection rate.  

\end{frame}


\section{Gaussian Processes and Epidemic Models}
\subsection{Introduction}
\begin{frame}
%% Gaussian Processes - Intro
	\frametitle{What are Gaussian Processes?}
	\begin{columns}[T] 
     \begin{column}[T]{5cm} 
     \begin{itemize}
     	\item Gaussian processes (GPs) are a generalisation of the multivariate Gaussian distribution to an infinite function space.  
     	\item Gaussian processes (GPs) are a popular machine learning tool for learning functions. 
   		\item One of the main uses in non-parametric regression. 
     \end{itemize}
     \end{column}
     \begin{column}[T]{5cm} 
     \centering
     \includegraphics[scale = 0.5]{GP_cover_2}
     \end{column}
     \end{columns}
     \nocite{Ras06}
\end{frame}

\subsection{Definition}
\begin{frame}
% Gaussian Processes - Definition
	\frametitle{Defintion}
	\begin{definition}
		A Gaussian process is a collection of random variables, any finite number of which have a joint Gaussian distribution.
	\end{definition}
	
	To specify a GP, we need to define the mean $m(\textbf{x})$ and covariance function $k(\textbf{x}, \textbf{x}')$. We write it as 
	
	$$
	f \sim \mathcal{GP}\big(m(\textbf{x}), \> k(\textbf{x}, \textbf{x}')\big)
	$$
	
	We can input our assumptions of the function through the covariance function. 

\end{frame}

\begin{frame}
%% Gaussian Processes - Covariance
	\frametitle{Covariance Functions}
	\centering
	\begin{figure}
		\includegraphics[width = \textwidth, height = 0.2\textheight]{sq_exp}
		\caption{Three draws from the square exponential covariance function.}
		\includegraphics[width = \textwidth, height = 0.2\textheight]{matern2}
		\caption{Three draws from the M\'atern covariance function.}
		\includegraphics[width = \textwidth, height = 0.2\textheight]{periodic}
		\caption{Three draws from the periodic covariance function.}
	\end{figure}
\end{frame}

\begin{frame}
%% Gaussian Processes - Covariance
	\frametitle{Covariance Functions}
	The covariance function we will use is the square exponential
	
	$$
	k(x, x'; \alpha, l) = \alpha^2\exp\Big\{\ -\frac{(x-x')^2}{l^2} \Big\}
	$$
	\vspace{2em}
	\centering
	\includegraphics[height = 0.5\textheight, width = \textwidth]{Covariance_Length_Scale} \\
	
\end{frame}




\subsection{Inference with GPs}
\begin{frame}
%% Gaussian Processes - Inference for Epidemics
	\frametitle{Inference for Epidemics with GPs}
	How can we use GPs for modelling the infection rate? 
	\begin{enumerate}
		\item Adopt a Bayesian Framework
		\item Put a GP prior on infection rate
		\item Use data augmentation to overcome intractability
		\item Develop efficient MCMC algorithm to explore posterior density 
	\end{enumerate}
	\nocite{Xu16}
\end{frame}

\section{Distance Dependent Model}
\begin{frame}
%% Distance Model - Setup
	\frametitle{Building the Model}
	We will model the spread of a disease where the infection rate is distance dependent. \\
	\vspace{5mm}
	The individuals will be fixed on a 2D plane. \\
	\vspace{5mm}
	An example of this is the spread of Foot and Mouth Disease or Avian Flu.\\
	\pause
	\vspace{5mm}
	For example, we could assume $i$ and $j$, $\beta_{i,j}$, is
	
	$$
	\beta_{i, j} = \beta_0\exp\{-\rho(i, j)\},
	$$
	where $\rho(i, j)$ is the distance between $i$ and $j$.
\end{frame}

\begin{frame}
%% Distance Model - Animation
	\frametitle{Animation}
	\centering
	%%\includemovie[poster, autoplay, repeat]{6cm}{6cm}{inf_animation.swf}
\end{frame}

\subsection{The Statistical Model}
\begin{frame}
%% Distance Model - Likelihood 
	\frametitle{The Statistical Model}
The likelihood function for this model is given by
	
\begin{align*}
\pi(\textbf{i}, \textbf{r}| \boldsymbol{\beta}, \lambda, \gamma) \propto &\exp\Big(- \sum\limits_{j=1}^n\sum\limits_{k=1}^N \beta_{j, k}\big((r_j \wedge i_k) - (i_j \wedge i_k)\big)\Big)\prod\limits_{j=1}^n\Big(\sum\limits_{k \in \mathcal{Y}_j} \beta_{k, j}\Big)  \\ 
&\times \frac{\gamma^{n\lambda}}{\Gamma(\lambda)^n}\prod\limits_{j=1}^n(r_j-i_j)^{\alpha-1}\exp\Big\{-\gamma\sum\limits_{j=1}^n (r_j-i_j)\Big\}. 
\end{align*}
We put the following GP prior distribution on $\beta$
$$
 \beta_{j, k} = \exp\big\{g(\rho(j, k))\big\}, \quad g \sim \mathcal{GP}(\boldsymbol{\mu}, \Sigma).
$$

We also put an exponential prior distribution on $\gamma$, the infectious period rate parameter. 

$$
\gamma \sim \hbox{Exp}(\nu).
$$
\nocite{Theo07}


\end{frame}

\begin{frame}
\frametitle{The Statistical Model}
The posterior density is given by
\begin{align*}
\hspace{-2em}\pi(g, \gamma |\textbf{i}, \textbf{r}, \lambda) & \propto  \> \mathcal{GP}(\textbf{g})\exp\Big(- \sum\limits_{j=1}^n\sum\limits_{k=1}^N \exp\big\{g(\rho(j, k))\big\}\big((r_j \wedge i_k) - (i_j \wedge i_k)\big)\Big)  \\ 
& \times \prod\limits_{j=1}^n\Big(\sum\limits_{k \in \mathcal{Y}_j} \exp\big\{g(\rho(j, k))\big\}\Big)\gamma^{n\lambda}\exp\Big\{-\gamma\sum\limits_{j=1}^n  (r_j-i_j)\Big\}\\
& \times  \gamma\exp\{-\nu\gamma\}. 
\end{align*}
We explore the posterior density using MCMC
\begin{itemize}
	\item Data Augmentation for infection times
	\item MH for $\boldsymbol{\beta}$ and infection times
	\item Gibbs sampler for $\gamma$
\end{itemize}
\end{frame}


\begin{frame}
%% Distance Model - Results
	\frametitle{Example}
	\centering
	\begin{figure}
		\includegraphics[width = \textwidth]{fixed_i_1}
	\caption{MCMC output when infection times are known.}
	\end{figure}

\end{frame}

\begin{frame}
%% Distance Model - Results
	\frametitle{Example}
	\centering
	\begin{figure}
		\includegraphics[width = \textwidth]{unknown_i_1}
		\caption{MCMC output when infection times are unknown.}
	\end{figure}
	
\end{frame}


\section{Improvements}
\subsection{Including More Covariates}
\begin{frame}
%% Improvements - Weighting
	\frametitle{Including More Covariates}
	So far we have only looked at including distance in our model and 
	we want to include other multiple inputs in our model. \\
	\vspace{1em}
	
	For example, in the case of avian flu we might want to include:
	\begin{itemize}
		\item Distance
		\item Number of birds on each farm
		\item Type of farm
		\item Type of bird on each farm
		\item Farms which are owned by the same group
	\end{itemize}

\end{frame}

\subsection{Efficiency}
\begin{frame}
%% Improvements - Size 
	\frametitle{Efficiency}
	There are two main obstacles with MCMC methods with GPs:\\
		\textbf{MCMC:}\\
	\begin{itemize}
		\item MCMC can be very slow and computationally intensive.
		\item We're looking to make the code more efficient. 
	\end{itemize}
	\textbf{The Covariance Matrix:}\\
	\begin{itemize}
		\item Inverting and decomposing are computationally expensive.
		\item Can we choose strategic input points?
		\item How can we implement sparsification methods?
	\end{itemize}



\end{frame}

\begin{frame}
%% Thanks and References 
\frametitle{Selected References}
\setbeamertemplate{bibliography item}[triangle]
\bibliographystyle{plain}
\bibliography{bibliography}
\end{frame}




\end{document}
